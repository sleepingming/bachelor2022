\chapter*{Введение}


Спутниковая навигация в настоящее время является незаменимой частью повседневной жизни человека. Она лежит в основе множества сфер деятельности, к которым мы уже так привыкли. Сейчас практически каждый смартфон имеет навигационный модуль и, пользуясь картами, вызывая такси или записывая трек своей тренировки, мы непременно обращаемся к спутниковой навигации. ГНСС обеспечивают работу логистических систем, транспорта, связывают предприятия сферы услуг с их клиентами и т.д. Глобальными навигационными системами являются системы позиционирования, такие, как NAVSTAR GPS (США), ГЛОНАСС (Россия), Galileo (Евросоюз) и Beidou (Китай)\cite{korogodin}. 

Спутниковый навигационный приёмник --- радиоприёмное устройство, позволяющее определить
географические координаты местоположения антенны приёмника, опираясь на данные о временных задержках прихода радиосигналов, излучаемых
спутниками навигационных систем. Назначением навигационного приёмника, как и всей навигационной системы является предоставление возможности позиционирования (определения местоположения объекта) и навигации потребителю. Приёмник получает сигналы от спутников ГНСС, обрабатывает их, выдавая затем цифровые сигналы, которые необходимы для определения координат, скорости и времени -- это и является решением навигационной задачи. Конечной целью навигационной задачи является получение оценки вектора состояния потребителя:
\begin{equation}
	\mathbf{\Pi} = |x\;y\;z\;t\;V_x\;V_y\;V_z|^T,
\end{equation}
где $\{x,y,z\}$ -- координаты потребителя в той или иной системе координат, $t$ -- время, $\{V_x,V_y,V_z\}$ -- составляющие вектора скорости.

Элементы вектора состояния $\mathbf{\Pi}$ недоступны измерениям непосредственно с помощью радиосредств. У принятого радиосигнала могут измеряться различные его параметры, например, временная задержка или доплеровское смещение частоты. Параметр радиосигнала, измеряемый в целях навигации называют радионавигационным (РНП), а соответствующий ему геометрический параметр -- навигационным (НП). Поэтому задержка сигнала $\tau$ и доплеровское смещение частоты $f_{д}$ являются радионавигационными параметрами, а соответствующие им дальность до объекта \textit{Д} и радиальная скорость сближения объектов $V_p = \dot{Д}$ служат навигационными параметрами, связь между которыми даётся соотношениями:
\begin{equation}
	\begin{split}
	\textit{Д} = c \tau, \\
	V_p = f_{д} \lambda,
	\end{split}
\end{equation}
где \textit{c} -- скорость света; $\lambda$ -- длина волны излучаемого НС сигнала.

В СРНС в силу большого расстояния между передающей и приёмной позициями фиксация моментов излучения и приёма сигнала не может выполняться в одой шкале времени. Время излучения сигнала с борта навигационного спутника определяется в бортовой шкале времени $t^{БШВ}$, а время приёма сигнала -- в шкале времени потребителя $t^{ШВП}$. В СРНС решается задача определения длительности интервала между моментами времени ($t^{БШВ}$ — момент
излучения некоторой фазы дальномерного кода с борта НС и $t^{ШВП}$ — момент приема той же фазы дальномерного кода у потребителя), заданные в различных шкалах. Для такого интервала времени используется термин \textit{псевдо задержка}. Если $Т_{пр}(t^{ШВП}_{пр})$ — значение времени на ШВП в момент времени
$t^{ШВП}_{пр}$, а $Т_{изл}(t^{БШВ}_{изл})$ — значение времени на БШВ в момент времени $t^{ШВП}_{пр}$, то значение псевдо задержки определяется соотношением:
\begin{equation}
	\breve{\tau}(t^{ШВП}_{пр})=Т_{пр}(t^{ШВП}_{пр})-Т_{изл}(t^{БШВ}).
\end{equation}
Определим линейную величину соотношением:
\begin{equation}
	\breve{Д}(t^{ШВП}_{пр})=с\breve{\tau}(t^{ШВП}_{пр}),
\end{equation}

Имея смысл дальности (расстояния), параметр $\breve{Д}$ не является дальностью в обычном понимании, т.е. дальностью между двумя точками в пространстве. Поэтому для $\breve{Д}$, определенному с использованием (3), (4), используется термин \textit{псевдо дальность}. Параметр
\begin{equation}
	\dot{\breve{Д}}(t^{ШВП}_{пр})=\breve{V}_i(t^{ШВП}_{пр}),
\end{equation}
называют \textit{псевдо скоростью}, а соответствующий ей параметр 
\begin{equation}
	\breve{f_{дi}}=\frac{\breve{V}_i(t^{ШВП}_{пр})}{\lambda_i}
\end{equation}
-- \textit{псевдо доплеровским смещением частоты.}

В результате первичной обработки радионавигационных сигналов в 
заданные моменты времени $t_k$ формируются оценки радионавигационных параметров (псевдозадержки $\hat{\breve{\tau}}_{i,k}$ и псевдодоплеровского смещения частоты $ \hat{\breve{f}}_{дi,k}$) в
общем случае для всех видимых спутников (i = $\overline{1,N}$). Данные оценки можно представить в виде
\begin{equation}
	\hat{\breve{\tau}}_{i,k}=\breve{\tau}_{i,k}+n_{\tau i,k},
\end{equation}
\begin{equation}
		\hat{\breve{f}}_{дi,k}=\breve{f}_{д i,k}+n_{f i,k},
\end{equation}
где $\breve{\tau}_{i}$, $\breve{f}_{д i}$ — истинные значения псевдозадержки и псевдодоплеровского смещения частоты; $n_{\tau i}$, $n_{f i}$ — погрешности их оценок, полученных на этапе первичной обработки сигналов. Оценки $\hat{\breve{\tau}}_{i,k}$ и $\hat{\breve{f}}_{дi,k}$ обычно называют \textit{сырыми измерениями}.

Таким образом, итоговая навигационная задача заключается в получении оценок вектора потребителя по имеющимся векторным вторичным наблюдениям при условии наличия информации о параметрах движения (координатах и векторах скорости) навигационных спутников, которая формируется в приёмнике потребителя в результате декодирования навигационного сообщения из принятых радиосигналов.

\textit{Дифференциальный режим навигации.} Приемник в реальном времени
обеспечивает декодирование корректирующих поправок, передаваемых в 
соответствии с форматом RTCM, и оценивание координат объекта с
точностью до десятков сантиметров. В технике \textbf{DGNSS} (англ. differencial GNSS) для увеличения точности позиционирования приемник дополнительно использует кодовые измерения от одной или нескольких базовых станций, расположенных на удалении до нескольких десятков километров.
При этом, как правило, данные передаются посредством дополнительного радиоканала или сигналов сотовой сети. 
В качестве \textit{базовых станций} выступают специальные неподвижные ГНСС приемники с точно известными координатами\cite{perov10glonass}.

Режим \textbf{RTK} (Real Time Kinematic, позиционирование движущегося объекта в реальном времени) является дальнейшим расширением режима дифференциальной навигации и позволяет на основе приема данных об измерениях псевдо дальности и псевдо фазы, выполняемых в базовом приемнике, в реальном времени оценивать текущие координаты приемника с точностью до единиц миллиметров в статическом режиме и до единиц сантиметров в динамическом режиме. В режиме \textit{RTK} данные от базовой станции передаются роверу по некоторому каналу связи в момент снятия измерений, что позволяет сразу получить точное навигационное решение.
Данный режим широко применяется при проведении наземных геодезических работ, т.к. позволяет выявить возможные проблемы сразу в момент съемки и внести оперативные исправления. 

Данные, передаваемые спутниками, навигационный приёмник получает в бинарном формате в виде пакетов. Для обработки полученных навигационных данных можно воспользоваться библиотекой RTKLIB – популярным программным обеспечением для постобработки GNSS с открытым исходным кодом. Данное программное обеспечение позволяет найти
решение навигационной задачи, то есть получить координаты объекта и высоту,
на которой он находится по входным данным, полученным из навигационного
приемника[2], за это отвечают модули RTKNAVI и RTKPLOT. Возможности
RTKNAVI позволяют наблюдать сигналы практически всех существующих навигационных
систем, а также осуществлять приём и обработку данных по различным протоколам, например, по NVS BINR, RTCM, BINEX, u-blox и другим.