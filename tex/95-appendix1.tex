\chapter{Структуры данных пакетов}
\label{cha:appendix6}
\begin{longtable}[c]{|lll|}
	\caption{Структура данных пакета 0x00F5} \label{tab:formatf5}\\
	\hline
	Название поля & Тип данных & Содержимое \\
	\hline
	time & double & Время измерений, мс (UTC) \\
	week & uint16 & Номер недели \\
	dGPSUTC & double & Разность времен GPS-UTC, мс \\
	dGLNUTC & double & Разность времен ГЛОНАСС-UTC, мс \\
	TimeScaleCorrection & int8 & Коррекция шкалы приемника, мс \\
	type & int8 & Тип сигнала \\
	SgnID & int8 & Номер спутника (с единицы) \\
	Lit & int8 & Частотная литера (для ГЛОНАСС) \\
	q & int8 & Уровень сигнала, дБГц \\
	phase & double & Псевдофаза сигнала, циклы \\
	range & double & Полная псевдодальность, мс \\
	doppler & double & Доплеровская частота, Гц \\
	flags & int8 & Флаги измерений \\
	\hline	
\end{longtable}
\newpage
\begin{longtable}[c]{|lll|}
	\caption{Структура данных пакета 0x00F7 для спутников GPS} \label{tab:formatf7gps}\\
	\hline
	Название поля & Тип данных & Содержимое \\
	\hline
	GnssType & uint8 & Тип спутниковой системы \\
	SatNum & uint8 & Номер спутника \\
	Crs & float & $C_{rc}$ , м \\
	Dn & float & $\Delta n$, рад/мс \\
	M0 & double &  $M_0$, рад \\
	Cuc & float & $C_{uc}$, рад\\
	e & double & е \\
	Cus & float & $C_{us}$, рад \\
	sqrtA & double & $\sqrt{A}, м^{1/2}$ \\
	toe & double & $t_{oe}$, мс \\
	Cic & float & $C_{ic}$, рад \\
	Omega0 & double & $\Omega_{0}$, рад \\
	Cis & float & $C_{is}$, рад \\
	i0 & double & $i_{0}$, рад \\
	Crc & float & $C_{rc}$, рад \\
	omega & double & $\omega$, рад \\
	OmegaDot & double & $\dot{\Omega}$, рад/мс \\
	iDot & double & $\dot{\Omega}$, рад/мс \\
	Tgd & float & $IDOT$, рад/мс \\
	toc & double & $T_{GD}$, мс \\
	af2 & float & $a_{f2},\;мс/мс^2$ \\
	af1 & float & $a_{f1},\;мс/мс$ \\
	af0 & float & $a_{f0},\;мс$ \\
	URA & uint16 &  \\
	IODE & uint16 &  \\
	IODC & uint16 &  \\
	codeL2 & uint16 & C/A OR P ON L2 \\
	L2P & uint16 & L2 P DATA FLAG \\
	week & uint16 & Номер недели \\
	\hline	
\end{longtable}
\newpage
\begin{longtable}[c]{|lll|}
	\caption{Структура данных пакета 0x00F7 для спутников ГЛОНАСС} \label{tab:formatf7glo}\\
	\hline
	Название поля & Тип данных & Содержимое \\
	\hline
	GnssType & uint8 & Тип спутниковой системы \\
	SatNum & uint8 & Номер спутника \\
	Lit & int8 & $x_{n}(t_b)$ , м \\
	X & double & $\Delta n$, рад/мс \\
	Y & double &  $M_0$, рад \\
	Z & double & $C_{uc}$, рад\\
	Vx & double & е \\
	Vy & double & $C_{us}$, рад \\
	Vz & double & $\sqrt{A}, м^{1/2}$ \\
	Ax & double & $t_{oe}$, мс \\
	Ay & double & $C_{ic}$, рад \\
	Az & double & $\Omega_{0}$, рад \\
	tb & double & $C_{is}$, рад \\
	gamman & float & $i_{0}$, рад \\
	taun & float & $C_{rc}$, рад \\
	En & uint16 & $\omega$, рад \\
	\hline	
\end{longtable}

\chapter{Разработка алгоритма}
\section{Программный модуль для обработки пакетов данных}
\label{cha:appendix1}
\lstinputlisting[language=C]{main.c}

\newpage
\section{Функция для расчёта контрольной суммы}
\label{cha:appendix2}

\lstinputlisting[language=C]{crc.c}

\chapter{Имплементация протокола в RTKLIB}

\section{Структура <<raw\_t>>}
\label{cha:appendix3}

\lstinputlisting[language=C]{raw.c}
\newpage
\section{Функция <<decode\_srns>>}
\label{cha:appendix4}

\lstinputlisting[language=C]{decodesrns.c}

\section{Функция <<input\_srns>>}
\label{cha:appendix5}

\lstinputlisting[language=C]{inputsrns.c}
%%% Local Variables: 
%%% mode: latex
%%% TeX-master: "rpz"
%%% End: 
