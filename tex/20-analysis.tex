\chapter{Структура RTKLIB, описание интерфейсов}
\label{cha:analysis}

RTKLIB состоит из программной библиотеки и
вспомогательных программ. Также включает в себя стандартные функции и алгоритмы для обработки ГНСС: функции спутниковых систем навигации, матричные и векторные функции, расчет времени и строковые функции, функции преобразования координат, трассировка, поддержка систем позиционирования, расчет эфемерид и множество других функций.

Состав RTKLIB:
\begin{itemize}
	\item RTKLAUNCH
	\item RTKNAVI
	\item STRSVR
	\item RTKPOST
	
	\item RTKCONV
	\item RTKPLOT
	\item RTKGET
	\item SRCTBLBROWS
	\begin{figure}[ht]
		\centering
		\includegraphics[keepaspectratio, width=0.8\textwidth]{inc/svg/modules/all}
		\caption{Демонстрация всех приложений библиотеки}
		\label{fig:all}
	\end{figure}
\end{itemize}



\section{RTKLAUNCH}
Данный модуль является первым приложением библиотеки и включает в себя ссылки на составные программы.

\begin{figure}[ht]
	\centering
	\includegraphics[keepaspectratio, width=0.8\textwidth]{inc/svg/modules/rtklaunch}
	\caption{Демонстрация окна RTKLAUNCH}
	\label{fig:rtklaunch}
\end{figure}

RTKLAUNCH представляет собой интерфейс для удобного переключения между остальными вспомогательными программами.


\section{RTKNAVI}

Этот модуль имеет способность осуществлять позиционирование в реальном времени. На вход подаются сырые наблюдения, сформированные приемником, а на выходе мы получаем навигационные определения и решение навигационной задачи по данным первычным наблюдениям. Также можно настраивать выходной поток данных и записывать лог-файлы, что очень полезно для позиционирования в реальном времени.


\begin{figure}[h!]
	\centering
	\includegraphics[keepaspectratio, width=0.7\textwidth]{inc/svg/modules/rtknavi}
	\caption{Демонстрация окна RTKNAVI}
	\label{fig:rtknavi}
\end{figure}
\begin{figure}[h!]
	\centering
	\includegraphics[keepaspectratio, width=0.89\textwidth]{inc/svg/modules/rtklibdataflow}
	\caption{Структурная схема работы RTKNAVI}
	\label{fig:rtklibdataflow}
\end{figure}
\begin{figure}[h!]
	\centering
	\includegraphics[keepaspectratio, width=1.04\textwidth]{inc/svg/modules/rtklibdatastruct}
	\caption{Структуры данных RTKNAVI}
	\label{fig:rtklibdatastruct}
\end{figure}

Существует множество опций также и для подачи входных данных. Можно обрабатывать данные различных типов. Ниже приведены поддерживаемые библиотекой типы данных и их краткое описание.
\begin{longtable}{|p{4cm}|p{12cm}|}
	\caption{Типы принимаемых данных и их описание} \label{tab:types}\\
	\hline
	Тип данных & Описание \\
	\hline
	Serial & Приём данных через последовательный порт (RS232C или USB) \\
	\hline
	TCP Client & Соединение с TCP сервером и приём данных по протоколу TCP \\
	\hline
	TCP Server & Принятие TCP соединения клиента и приём данных по протоколу TCP. \\
	\hline
	NTRIP Client & Соединение с кастером NTRIP и приём данных по протоколу NTRIP. \\
	\hline
	File & Приём данных из лог-файла \\
	\hline
	FTP & Приём данных после загрузки файла с помощью FTP \\
	\hline
	HTTP & Приём данных после загрузки файла с помощью HTTP \\
	\hline
\end{longtable}
\FloatBarrier

Описание структур данных пакетов 0x00F5 и 0x00F7 приведены в приложении ~\ref{cha:appendix6}.

Также библиотека поддерживает обработку данных различных форматов и по различным протоколам. Ниже приведён список поддерживаемых форматов и их краткое описание.

\begin{longtable}{|p{4cm}|p{12cm}|}
	\caption{Форматы приёма данных и их описание} \label{tab:format}\\
	\hline
	Формат данных & Описание \\
	\hline
	RTCM2 & RTCM 2.3 \\
	\hline
	RTCM3  & RTCM 3.0, 3.1 (с поправкой 1‐5) и 3.2 \\
	\hline
	NovAtel OEM6 & Двоичный формат NovAtel OEM4/V/6 и OEMStar  \\
	\hline
	NovAtel OEM3  & Двоичный формат NovAtel OEM3 (Millennium) \\
	\hline
	u‐blox  & Двоичный формат u‐blox LEA‐4T, 5T и 6T  \\
	\hline
	Superstar II  & Двоичный формат NovAtel Superstar II  \\
	\hline
	\end{longtable}
	\newpage
	\begin{longtable}{|p{4cm}|p{12cm}|}
	\multicolumn{2}{r}{... продолжение таблицы \ref{tab:format}}\\ [1em] 
	\hline
	Hemisphere & Двоичный формат Hemisphere Crescent/Eclipse  \\
	\hline
	SkyTraq  & Двоичный формат SkyTraq S1315F  \\
	\hline
	GW10 &  Двоичный формат Furuno GW‐10‐II/III  \\
	\hline
	Javad  & Двоичный формат JAVAD GRIL/GREIS  \\
	\hline
	NVS BINR & NVS NV08C BINR \\
	\hline
	BINEX  & BINEX format (поддерживает только big‐endian, forward, regular CRC) \\
	\hline
	SP3  & SP3 точные эфемериды \\
	\hline
\end{longtable}

Основной задачей данной работы является, как раз, пополнение данного списка протоколом SRNS.

\subsection{Обработка файла по протоколу NVS BINR}
Для проверки работоспособности RTKNAVI можем подать на вход модуля двоичный лог-файл с наблюдениями, сформированными навигационным приёмником по протоколу NVS BINR. В результате мы должны получить решение навигационной задачи и таблицы, отображающие различную информацию, которая может быть полезна для нашего анализа: эфемериды, сырые данные, параметры сигналов, принимаемые пакеты и т.д.
\begin{figure}[h!]
	\centering
	\includegraphics[keepaspectratio, width=0.7\textwidth]{inc/svg/modules/rtknavitest}
	\caption{Расположение спутников}
	\label{fig:rtknavitest}
\end{figure}
\begin{figure}[h!]
	\centering
	\includegraphics[keepaspectratio, width=0.7\textwidth]{inc/svg/modules/rtknaviobs}
	\caption{Данные наблюдений}
	\label{fig:rtknaviobs}
\end{figure}
\begin{figure}[h!]
	\centering
	\includegraphics[keepaspectratio, width=0.7\textwidth]{inc/svg/modules/rtknavinavgps}
	\caption{Параметры спутников GPS}
	\label{fig:rtknavinavgps}
\end{figure}
\newpage
\begin{figure}[h!]
	\centering
	\includegraphics[keepaspectratio, width=0.7\textwidth]{inc/svg/modules/rtknavinavglo}
	\caption{Параметры спутников ГЛОНАСС}
	\label{fig:rtknavinavglo}
\end{figure}
\section{STRSVR}
Для относительного позиционирования, такого как RTK-GPS/GNSS, приёмник ровера и приёмник базовой станции обычно размещаются на разных участках. В других случаях пользователь может использовать результат определения местоположения в другом месте, удалённом от этих приёмников. Для соединения этих мест друг с другом, пользователь должен установить каналы передачи данных. Данное приложение представляет собой утилиту сервера связи для упрощения настройки каналов связи. С помощью STRSVR пользователь может настраивать поток входных и выходных данных по этим каналам связи. Также данный модуль имеет функцию ретрансляции для разделения потока данных для определения местоположения в реальном времени с помощью RTKNAVI.

Например, чтобы получить наблюдения удалённой базовой станции на приёмнике ровера и получить решение RTK-GPS, пользователь может разместить удалённый компьютер с установленным STRSVR и, подключившись к приёмнику базовой станции, настроить STRSVR для отправки данных роверу.


Ниже приведены примеры совместного использования приложений RTKNAVI и STRSVR.
\begin{figure}[ht]
	\centering
	\includegraphics[keepaspectratio, width=0.9\textwidth]{inc/svg/modules/strsvr1}
	\caption{Single-point позиционирование и вывод решений в файл}
	\label{fig:strsvr1}
\end{figure}
\begin{figure}[ht]
	\centering
	\includegraphics[keepaspectratio, width=0.9\textwidth]{inc/svg/modules/strsvr2}
	\caption{Single-point позиционирование, вывод решений на устройство, вывод лог-данных в файл}
	\label{fig:strsvr2}
\end{figure}
\begin{figure}[h!]
	\centering
	\includegraphics[keepaspectratio, width=0.9\textwidth]{inc/svg/modules/strsvr3}
	\caption{RTK-GPS/GNSS позиционирование, ввод данных с ровера и базовой станции, вывод решений в файл}
	\label{fig:strsvr3}
\end{figure}
\FloatBarrier
\begin{figure}[ht]
	\centering
	\includegraphics[keepaspectratio, width=0.9\textwidth]{inc/svg/modules/strsvrmain}
	\caption{Демонстрация главного окна STRSVR}
	\label{fig:strsvrmain}
\end{figure}

С помощью пункта меню <<(0) Input>> можно выбирать тип входного потока. Поддерживаемые типы:
\begin{itemize}
	\item[$\triangleright$] Serial
	\item[$\triangleright$] TCP-клиент
	\item[$\triangleright$] TCP-сервер
	\item[$\triangleright$] NTRIP-клиент
	\item[$\triangleright$] Файл
	\item[$\triangleright$] FTP
	\item[$\triangleright$] HTTP
\end{itemize}
Также есть возможность задать параметры потока или команду запуска/выключения и входные потоки для RTKNAVI.

Настройка выходных потоков производится с помощью пунктов (1), (2) или (3). Для выходных потоков такие же настройки, как и для лог-потоков в RTKNAVI.

\section{RTKPOST}

Данное приложение способно осуществлять постобработку. RTKPOST получает на вход файлы наблюдений и навигационных сообщений (GPS, ГЛОНАСС, Galileo, QZSS, BeiDou и SBAS) в формате RINEX (2.10, 2.11,
2.12, 3.00, 3.01, 3.02) и может получать решение навигационной задачи с помощью различных техник позиционирования, включая Single‐point, DGPS/DGNSS, Kinematic, Static, PPP‐Kinematic и PPP‐Static.
\begin{figure}[ht]
	\centering
	\includegraphics[keepaspectratio, width=0.8\textwidth]{inc/svg/modules/rtkpost}
	\caption{Демонстрация главного окна RTKPOST}
	\label{fig:rtkpost}
\end{figure}

С помощью меню <<Options...>> можно настраивать параметры обработки и параметры позиционирования для RTKNAVI и RTKPOST. Также можно установить время начала и время окончания наблюдения.
\begin{figure}[ht]
	\centering
	\includegraphics[keepaspectratio, width=0.8\textwidth]{inc/svg/modules/rtkpostsolv}
	\caption{Пример решения навигационной задачи с помощью RTKPOST}
	\label{fig:rtkpostsolv}
\end{figure}

\section{RTKCONV}
RINEX -- это стандартный формат данных GPS/GNSS, поддерживаемый многими
приемниками или программными обеспечениями для постобработки GPS/GNSS. RTKPOST также может обрабатывать файлы RINEX в качестве входных данных, но перед этим необходима некоторая подготовка файлов к обработке. Возможность такой подготовки предоставляет приложение RTKCONV, который преобразует сырые данные приёмника, RTCM и BINEX сообщения в RINEX OBS (данные наблюдений) и RINEX NAV (навигационные сообщения GNSS). Также RTKCONV может извлекать сообщения SBAS из сырых данных, полученных с приёмника, и выводить их в лог-файл SBAS.
\begin{figure}[ht]
	\centering
	\includegraphics[keepaspectratio, width=0.8\textwidth]{inc/svg/modules/rtkconv}
	\caption{Демонстрация главного окна RTKCONV}
	\label{fig:rtkconv}
\end{figure}

\section{RTKPLOT}
Данное приложение осуществляет построение решений позиционирования с помощью RTKPOST и RTKNAVI, используя графический пользовательский интерфейс. RTKPLOT также может принимать файлы или потоки по стандарту NMEA 0183 для создания графика решения. Есть возможность отображать E/N/U компоненты положения, скорости и ускорения приёмника. 

С помощью пункта меню <<File $\Rightarrow$ Open map Image>> можно выбрать изображение карты в формате JPEG и сделать его фоном решения в разделе <<Gnd Trk>>. Начиная с версии 2.4.2 в RTKPLOT можно отображать решения НЗ используя Google Earth.

Также RTKPLOT имеет функцию отображения видимых спутников различных навигационных систем, их орбиты и прочие параметры.

\begin{figure}[h!]
	\centering
	\includegraphics[keepaspectratio, width=0.5\textwidth]{inc/svg/modules/rtkplot}
	\caption{Пример решения навигационной задачи в RTKPLOT}
	\label{fig:rtkplot}
\end{figure}
\begin{figure}[h!]
	\centering
	\includegraphics[keepaspectratio, width=0.5\textwidth]{inc/svg/modules/rtkplotmap}
	\caption{Пример вставки изображения карты на фоне решения в RTKPLOT}
	\label{fig:rtkplotmap}
\end{figure}
\begin{figure}[h!]
	\centering
	\includegraphics[keepaspectratio, width=0.5\textwidth]{inc/svg/modules/rtkplotgoogle}
	\caption{Пример отображения решения с помощью Google Earth}
	\label{fig:rtkplotgoogle}
\end{figure}
\clearpage
\begin{figure}[h!]
	\centering
	\includegraphics[keepaspectratio, width=0.6\textwidth]{inc/svg/modules/rtkplotsat}
	\caption{Демонстрация отображения видимых спутников}
	\label{fig:rtkplotsat}
\end{figure}
\section{RTKGET}
Для анализа техники позиционирования PPP пользователю вероятнее всего придётся загрузить IGS, для получения точной информации об орбите спутника и часах. В других случаях может потребоваться загрузить данные наблюдений сети CORS из архива данных GNSS через интернет. Утилита RTKGET предоставляет удобный способ загрузки этих данных для дальнейшего использования в целях обработки навигационных данных.
\begin{figure}[ht]
	\centering
	\includegraphics[keepaspectratio, width=0.5\textwidth]{inc/svg/modules/rtkget}
	\caption{Демонстрация отображения видимых спутников}
	\label{fig:rtkget}
\end{figure}

\section{SRCTBLBROWS}
Данное приложение представляет собой NTRIP Browser. NTRIP -- это протокол связи для обмена данными GPS/GNSS, такими как: сырые наблюдения приёмника, эфемериды, поправки для DGPS или RTK‐GPS. NTRIP задаёт формат таблицы так называемой <<исходной таблицы>>, которая представляет список данных, предоставляемых серверами NTRIP. RTKLIB включает в себя простой браузер для исходных таблиц NTRIP\cite{rtklib}.
\begin{figure}[ht]
	\centering
	\includegraphics[keepaspectratio, width=0.8\textwidth]{inc/svg/modules/ntrip}
	\caption{Демонстрация главного окна SRCTBLBROWS}
	\label{fig:ntrip}
\end{figure}


