\chapter*{Заключение} % заключение к отчёту
В ходе данной выпускной квалификационной работы был разработан программный модуль для приёма навигационных данных по протоколу SRNS, а также осуществлено его внедрение в программное обеспечение RTKLIB.

Сначала отдельно был разработан алгоритм, способный обрабатывать пакеты данных, принятых с навигационного приёмника и выделять из них нужную информацию. Далее было осуществлено предварительное тестирование данного алгоритма с помощью простого вывода порядковых номеров пакетов, а затем была реализована функция для подсчёта контрольной суммы с помощью алгоритма CRC32. Расчёт контрольной суммы позволил нам удостовериться в корректности принятых пакетов и позволил нам перейти к следующему этапу: имплементация протокола в библиотеку RTKLIB.

Для внедрения алгоритма в библиотеку необходимо было изучить её исходные коды, в основном -- коды обрабатывающих файлов. Был осуществлён вывод <<SRNS>> на форму приложения для возможности выбора данного протокола в настройках входных данных. Затем к простому выводу пункта меню <<SRNS>> был подкреплён функционал -- теперь при выборе данного протокола вызывается соответствующая обрабатывающая функция, которая впоследствии совершает процесс декодирования принятых данных.

Испытания программного модуля были проведены как с пакетами, записанными в файл, так и в реальном времени, подключаясь по TCP/IP к навигационному приёмнику. Для тестирования в реальном времени использовалась конфигурация из навигационного приёмника и имитатора сигналов ГНСС (Rohde\&Schwarz SMBV100A). Далее сравнивались расположения спутников, отображаемые имитатором, с расположениями, полученными с помощью RTKLIB. Результаты сошлись, что можно наблюдать на рисунках 3.14 и 3.12. Также с помощью онлайн-программы GNSS Planning были получены рисунки с расположениями спутников и их орбитами 19 февраля 2014 года в 06:00 часов относительно г. Москва (исходя из настроек имитатора) и на рисунках 3.19 и 3.14 можем наблюдать, что положения спутников сошлись.

\phantom{ \cite{GNSSPlanningOnline} }
\phantom{ \cite{wireless} }


%%% Local Variables: 
%%% mode: latex
%%% TeX-master: "rpz"
%%% End: 

%\chapter*{Благодарности} % заключение к отчёту
%Хочу выразить огромную благодарность Потехину Роберту Наилевичу за участие в данной работе. Роберт Наилевич оказал необъемлемое количество помощи в тестировании работы разработанного программного модуля, а также систематически осуществлял психологическую поддержку не только во время разработки, но и в процессе всего четырёхлетнего обучения по программе бакалавриата.